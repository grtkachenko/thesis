\chapter{Применение}
\section{Решение задачи обратного геокодирования}
Задача обратного геокодирования (reverse geocoding) состоит в
определении адреса по координатам точки.

Входными данными к этой задаче обычно является множество объектов
с известными географическими адресами. Поиск страны и некоторых других частей
адреса осуществялется за счет локализации в подразбиении карты мира на страны, 
страны на регионы и так далее. Для решения данной задачи известно множество методов.
Локализация на уровне улиц происходит иным образом. Определение улицы, рядом с 
которой расположена точка, осуществляется за счет поиска ближайшей улицы или дома (рис. \ref{rgeocoding}).

Эти данные представимы в виде отрезков, поэтому можно предобработать данные карты,
поместив геометрию карты в структуру для быстрого поиска ближайших отрезков, и запомнив
соответствие между геометрическими данными и их метаинформацией.

В результате, с помощью реализованного алгритма можно быстро решать задачу обратного геокодирования.

\drawfigure{rgeocoding}{Ближайший объект}{rgeocoding}

\FloatBarrier
\section{Решение задач интерполяции}
Быстрый поиск ближайших отрезков часто используется при интерполяции
функций, заданных на линейных объектах. Например, можно
восстанавливать высотную модель по изолиниям.

Также, зная расстояние до ближайшей линии, можно варьировать параметры
интерполяции. В работе \cite{NGRID} описано применение кампанией Транзас алгоритма
поиска ближайшего отрезка для восстановления рельефа по
картографическим данным. Они накладывают на известные данные шумы,
зависящие от расстояния до изолиний. Вблизи изолиний используется
высокочастотный шум низкой амплитуды, симулирующий мелкие неровности
рельефа, тогда как на удалении от них используется низкочастотный шум
большей амплитуды, позволяющий получать холмистую местность. Этот
подход позволяет по картографическим данным получать реалистичный
рельеф.

В этой же работе описано использование информации о ближайших отрезках
для простого текстурирования рельефа. В зависимости от расстояния до
линейных объектов применяются различные текстуры, которые плавно
переходят друг в друга. Например, в зависимости от расстояния до
береговой линии, сначала может идти текстура песка, а потом травяной растительности.

\FloatBarrier
