\chapter{Применение}
\section{Решение задачи обратного геокодирования}
Задача обратного геокодирования (reverse geocoding) состоит в
определении адреса по координатам точки.

Входными данными к этой задаче обычно является множество объектов
с известными географическими адресами. Поиск страны и некоторых других частей
адреса осуществялется за счет локализации в подразбиении карты мира на страны, 
страны на регионы и так далее. Для решения данной задачи известно множество методов.
Локализация на уровне улиц происходит иным образом. Определение улицы, рядом с 
которой расположена точка, осуществляется за счет поиска ближайшей улицы или дома (рис. \ref{rgeocoding}).

Эти данные представимы в виде отрезков, поэтому можно предобработать данные карты,
поместив геометрию карты в структуру для быстрого поиска ближайших отрезков, и запомнив
соответствие между геометрическими данными и их метаинформацией.

В результате, с помощью реализованного алгритма можно быстро решать задачу обратного геокодирования.

\drawfigure{rgeocoding}{Ближайший объект}{rgeocoding}

\FloatBarrier
