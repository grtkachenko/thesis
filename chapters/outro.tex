\startconclusionpage

Как было показано в данной работе, задача определения кратчайшего
расстояния до произвольной конфигурации отрезков на плоскости представляет
достаточный практический интерес. Были подробно рассмотрены применяемые
на практике подходы. Предложенный подход не только позволяет искать
ближайшие отрезки быстрее общепринятых методов, но еще и гораздо проще в
реализации.

В данный момент большой практический интерес представляет ускорение
построения lower envelope за счет отказа от точной арифметики в пользу более
быстрых альтернативных техник, таких как, например, adaptive precision
arithmetic \cite{APREC}. Это позволит сократить время на построение, как минимум в
десять раз. Также интересным было бы разработать технику отсечения
заведомо далеких отрезков перед построением lower envelope .
Из недостатков данного подхода стоит отметить лишь большее, по
сравнению с другими рассмотренными методами, потребление памяти. Но для
многих задач данные затраты оказываются вполне приемлемыми.
Также хотелось бы отметить, что хоть данный метод применялся для
отрезков, он легко обобщается и для других классов объектов. К тому же
данный подход применим и для других классов задач, таких как поиск наиболее
удаленного отрезка (furthest segment problem) и поиск k ближайших соседей
(k-Nearest Neighbor Search). Также данный метод не требует построения
пересечений отрезков для своей работы, в отличии от других рассмотренных
методов.
