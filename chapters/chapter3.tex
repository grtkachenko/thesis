%-*-coding: utf-8-*-
\chapter{Сравнение с другими точными решениями}
\section{Модификация n-grid для точного решения}

В работе \cite{NGRID} была предложена структура данных n-grid, позволяющая
эффективно решать задачу поиска ближайшего отрезка. Автором этой работы
было предложено несколько алгоритмов фильтрации не ближайших отрезков из
ячеек сети. Но предложенные алгоритмы или осуществляют очень грубую
фильтрацию, оставляя при этом много лишних отрезков и не ухудшая точность
поиска ближайшего отрезка, или переходят к неточному решению задачи
поиска ближайшего отрезка, отфильтровывая, возможно, ближайшие отрезки.
Для этой структуры данных мной была применена та же схема
фильтрации, что и в квадродереве. После этого незначительного изменения
данная структура данных позволила решать задачу поиска ближайшего отрезка
точно и оптимально, в смысле количества отрезков, которое нужно
рассматривать перебором

\section{Сравнительные испытания}

В сравнении будут участвовать:
\begin{itemize}
\item SVD (реализация из библиотеки CGAL);
\item модифицированный n-grid (модифицированная мною, реализация, используемая в компании
Транзас);
\item квадродерево (моя реализация).
\end{itemize}
Все испытания проводились на случайно генерируемых данных. Для
генерации данных бралась прямоугольная область пространства, в которой из
случайной точки этой области начинала генерироваться случайная цепь.
Случайная цепь характеризуется максимальной и минимальной длинами шага,
а также максимальным и минимальным углами поворота. При выходе цепи за
пределы области начинается генерация новой цепи. В результате получается
нечто, напоминающее картографические данные.

Для определения скорости поиска ближайшего отрезка производится один
миллион случайных запросов, по которым усредняется время поиска.

В табл. \ref{query_time}, табл. \ref{preproc_time}, табл. \ref{mem_usage}, табл. \ref{max_mem_usage}
приведены усредненные результаты сравнительных испытаний.

\begin{table}[ht]
\centering
\renewcommand{\arraystretch}{1.1}
\captionbox{Скорость поиска ближайшего отрезка в миллисекундах
\label{query_time}
}{
\begin{tabular*}{1.0\textwidth}{@{\extracolsep{\fill}} |l|r|r|r|r|}
\hline
Алгоритм/количество отрезков & 100 & 1000 & 10000 & 100000 \\
\hline
quadtree & 0.00152 & 0.00148 & 0.00165 & 0.00268 \\
n-grid   & 0.00172 & 0.00165 & 0.00186 & 0.00373 \\
SVD      & 0.04676 & 0.04495 & 0.07017 & 0.08502 \\
\hline

\end{tabular*}}
\end{table}

\begin{table}[ht]
\centering
\renewcommand{\arraystretch}{1.1}
\captionbox{Время препроцессирования в секундах
\label{preproc_time}
}{
\begin{tabular*}{1.0\textwidth}{@{\extracolsep{\fill}} |l|r|r|r|r|}

\hline
Алгоритм/количество отрезков & 100 & 1000 & 10000 & 100000 \\
\hline
quadtree & 1.646 & 26.524 &  482.523 &  7027.799 \\
n-grid   & 6.716 & 86.304 & 1917.237 & 25193.771 \\
SVD      & 0.016 &  0.226 &    5.093 &   106.752 \\
\hline

\end{tabular*}}
\end{table}

\begin{table}[ht]
\centering
\renewcommand{\arraystretch}{1.1}
\captionbox{Количество памяти, занимаемой структурой, в мегабайтах
\label{mem_usage}
}{
\begin{tabular*}{1.0\textwidth}{@{\extracolsep{\fill}} |l|r|r|r|r|}

\hline
Алгоритм/количество отрезков & 100 & 1000 & 10000 & 100000 \\
\hline
quadtree & 1.154 & 2.988 & 19.386 & 184.775 \\
n-grid   & 1.150 & 2.061 &  9.267 &  65.876 \\
SVD      & 0.223 & 1.065 &  5.982 &  53.277 \\
\hline
\end{tabular*}}
\end{table}

\begin{table}[ht]
\centering
\renewcommand{\arraystretch}{1.1}
\captionbox{
Максимальное количество памяти, потребляемое при построении в мегабайтах
\label{max_mem_usage}
}{
\begin{tabular*}{1.0\textwidth}{@{\extracolsep{\fill}} |l|r|r|r|r|}

\hline
Алгоритм/количество отрезков & 100 & 1000 & 10000 & 100000 \\
\hline
quadtree & 1.249 & 4.010 &  28.751 &  266.834 \\
n-grid   & 1.659 & 7.387 & 174.917 & 2716.963 \\
SVD      & 0.223 & 1.065 &   5.982 &   53.277 \\
\hline

\end{tabular*}}
\end{table}

\FloatBarrier
Из результатов сравнения заметно, что многоуровневая сеть и
квадродерево работают на порядок быстрее, чем диаграмма Вороного для отрезков. Время, затрачиваемое на
препроцессинг, и потребление памяти больше, но это цена за скорости работы самой структуры.
Хоть квадродерево и занимает больше памяти, чем многоуровневая сеть, но
оно дает выигрыш в скорости поиска ближайшего отрезка и потребляет гораздо 
меньше памяти во время построения.

В табл. \ref{qt_char} приведены усредненные характеристики квадродеревьев,
полученных при испытаниях.


\begin{table}[ht]
\centering
\renewcommand{\arraystretch}{1.1}
\captionbox{Характеристики квадродеревьев\label{qt_char}}{
\begin{tabular*}{1.0\textwidth}{@{\extracolsep{\fill}} |l|r|r|r|r|}

\hline
Характеристика/количество отрезков & 100 & 1000 & 10000 & 100000 \\
\hline
Количество ячеек                & 47    & 646    & 12963    & 203109 \\
Среднее число отрезков в ячейке & 13.74 &  12.88 &    12.64 &     13.39 \\
Максимальная глубина            &  4    &   7.25 &    11    &     12 \\
\hline

\end{tabular*}}
\end{table}

\FloatBarrier
