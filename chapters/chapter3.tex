%-*-coding: utf-8-*-
\chapter{Сравнение с другими точными решениями}
\section{Модификация n-grid для точного решения}

В работе \cite{NGRID} была предложена структура данных n-grid, позволяющая
эффективно решать задачу поиска ближайшего отрезка. Автором этой работы
было предложено несколько алгоритмов фильтрации не ближайших отрезков из
ячеек сети. Но предложенные алгоритмы или осуществляют очень грубую
фильтрацию, оставляя при этом много лишних отрезков и не ухудшая точность
поиска ближайшего отрезка, или переходят к примерному решению задачи
поиска ближайшего отрезка, отфильтровывая возможно ближайшие отрезки.
Для этой структуры данных мной была применена та же схема
фильтрации, что и в квадродереве. После этого незначительного измения
данная структура данных позволила решать задачу поиска ближайшего отрезка
точно и оптимально, в смысле количества отрезков, которое нужно
рассматривать перебором

\section{Сравнительные испытания}

В сравнении будут участвовать:
\begin{itemize}
\item SVD (реализация из библиотеки CGAL);
\item модифицированный n-grid (модифицированная мною, реализация, используемая в компании
Транзас);
\item квадродерево (моя реализация).
\end{itemize}
Все испытания проводились на случайно генерируемых данных. Для
генерации данных бралась прямоугольная область пространства, в которой из
случайной точки этой области начинала генерироваться случайная цепь.
Случайная цепь характеризуется максимальной и минимальной длинами шага,
а также максимальным и минимальным углами поворота. При выходе цепи за
пределы области начинается генерация новой цепи. В результате получается
нечто, напоминающее картографические данные.

Для определения скорости поиска ближайшего отрезка производятся десять
тысяч случайных запросов.

В таблицах 1-4 приведены усредненные результаты сравнительных
испытаний.

\begin{fasttable}{%
Скорость поиска ближайшего отрезка в миллисекундах}{%
query_time}{|r|r|r|r|}
\hline
Алгоритм количество отрезков & 100 & 1000 & 10000 \\
\hline
n-grid   & 0,002725 & 0,002325 & 0,00198 \\
SVD      & 0,049175 & 0,047175 & 0,07255 \\
quadtree & 0,002375 & 0,002375 & 0,00195 \\
\hline
\end{fasttable}
%\begin{fasttable}{шласаша}{labeeel}{|c|c|c|}
%asdasdas & asdasd & asdasd
%\end{fasttable}

Таблица 1 Скорость поиска ближайшего отрезка в миллисекундах
Алгоритм количество отрезков
n-grid
SVD
quadtree
100
6,6535
0,0115
1,6375
1000
87,5088
0,2425
26,6068
10000
2256,4220
5,2418
488,0683
Таблица 2 Время препроцессирования в секундах
Алгоритм\\количество отрезков
n-grid
SVD
quadtree
100
1,2233
0,2200
2,0423
1000
2,0783
1,0590
12,5915
10000
9,4328
5,9438
265,6437
Таблица 3 Количество памяти, занимаемой структурой, в мегабайтах
Алгоритм\\количество отрезков
n-grid
SVD
quadtree
100
1,6870
0,2200
2,0423
1000
8,7105
1,0590
12,5915
10000
175,3023
5,9438
265,6437
Таблица 4 Максимальное количество памяти, потребляемое при построении в мегабайтах
Из результатов сравнения заметно, что многоуровневая сеть и
квадродерево работают на порядок быстрее, чем SVD. Время, затрачиваемое на
препроцессинг, и потребление памяти больше, но это цена за скорость работы.
Хоть квадродерево и потребляет больше памяти, чем многоуровневая сеть, но
при неравномерном распределении вершин диаграммы Вороного на плоскости
оно дает существенно лучшую скорость поиска ближайшего отрезка.
В таблице 5 приведены усредненные характеристики квадродеревьев,
полученных при испытаниях.
Характеристика\\количество отрезков
Количество ячеек
Среднее число отрезков в ячейке
Максимальная глубина
100
44
9,3
4
1000
596
12,5
7,25
Таблица 5 Усредненные характеристики квадродерева
