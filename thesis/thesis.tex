% -*-coding: utf-8-*-
% This is an AMS-LaTeX v. 1.2 File.

\documentclass{report}

%\usepackage{pscyr}
%\renewcommand{\rmdefault}{fjn}
%\renewcommand{\ttdefault}{fcr}

%\usepackage{showkeys}
\usepackage[T2A]{fontenc}
\usepackage[utf8x]{inputenc}
\usepackage[english,russian]{babel}
\usepackage{expdlist}
\usepackage[pdftex]{graphicx}
\usepackage{amsmath}
\usepackage{pbox}
\usepackage{amssymb}
\usepackage{amsthm}
\usepackage{array}
\usepackage{amsfonts}
\usepackage{xspace}
\usepackage{algorithm}
\usepackage{algorithmicx}
\usepackage{amsxtra} 
\usepackage{sty/dbl12}
\usepackage{srcltx}
\usepackage{epsfig}
\usepackage{varwidth}
\usepackage{verbatim}
\usepackage{sty/rac}
\usepackage{algpseudocode}
%\usepackage[russian]{sty/ralg}
\usepackage{listings}
\usepackage{placeins}
%\usepackage{caption}
%\usepackage{floatrow}
\usepackage{caption}
\captionsetup[table]{position=t,justification=raggedright,slc=off}
\makeatletter
\def\BState{\State\hskip-\ALG@thistlm}
\makeatother

%\usepackage[
%    top    = 2.00cm,
%    bottom = 2.00cm,
%    left   = 3.00cm,
%    right  = 1.50cm]{geometry}
\hoffset = -10mm
\voffset = -20mm
\textheight = 230mm
\textwidth = 165mm

%%%%%%%%%%%%%%%%%%%%%%%%%%%%%%%%%%%%%%%%%%%%%%%%%%%%%%%%%%%%%%%%%%%%%%%%%%%%%%

% Redefine margins and other page formatting

%\setlength{\oddsidemargin}{0.5in}

% Various theorem environments. All of the following have the same numbering
% system as theorem.

\theoremstyle{plain}
\newtheorem{theorem}{Теорема}
\newtheorem{prop}[theorem]{Утверждение}
\newtheorem{corollary}[theorem]{Следствие}
\newtheorem{lemma}[theorem]{Лемма}
\newtheorem{question}[theorem]{Вопрос}
\newtheorem{conjecture}[theorem]{Гипотеза}
\newtheorem{assumption}[theorem]{Предположение}

\theoremstyle{definition}
\newtheorem{definition}[theorem]{Определение}
\newtheorem{notation}[theorem]{Обозначение}
\newtheorem{condition}[theorem]{Условие}
\newtheorem{example}[theorem]{Пример}
\newtheorem{algo}[theorem]{Алгоритм}
%\newtheorem{introduction}[theorem]{Introduction}

\floatname{algorithm}{Алгоритм}

\algnewcommand\algorithmicand{\textbf{and}\xspace}
\algnewcommand\algorithmicor{\textbf{or}\xspace}
\algnewcommand\algorithmicnot{\textbf{not}\xspace}
\algnewcommand\algorithmictrue{\textbf{true}}
\algnewcommand\algorithmicfalse{\textbf{false}}
\algtext*{EndWhile} % Remove "end while" text
\algtext*{EndIf} % Remove "end if" text
\algtext*{EndFor} % Remove "end for" text
\algtext*{EndProcedure} % Remove "end for" text

\renewcommand{\proof}{\\\textbf{Доказательство.}~}

%\def\startprog{\begin{lstlisting}[language=Java,basicstyle=\normalsize\ttfamily]}

%\theoremstyle{remark}
%\newtheorem{remark}[theorem]{Remark}
%\include{header}
%%%%%%%%%%%%%%%%%%%%%%%%%%%%%%%%%%%%%%%%%%%%%%%%%%%%%%%%%%%%%%%%%%%%%%%%%%%%%%%

\numberwithin{theorem}{chapter}        % Numbers theorems "x.y" where x
                                        % is the section number, y is the
                                        % theorem number

%\renewcommand{\thetheorem}{\arabic{chapter}.\arabic{theorem}}

%\makeatletter                          % This sequence of commands will
%\let\c@equation\c@theorem              % incorporate equation numbering
%\makeatother                           % into the theorem numbering scheme

%\renewcommand{\theenumi}{(\roman{enumi})}

%%%%%%%%%%%%%%%%%%%%%%%%%%%%%%%%%%%%%%%%%%%%%%%%%%%%%%%%%%%%%%%%%%%%%%%%%%%%%%


%%%%%%%%%%%%%%%%%%%%%%%%%%%%%%%%%%%%%%%%%%%%%%%%%%%%%%%%%%%%%%%%%%%%%%%%%%%%%%%

%This command creates a box marked ``To Do'' around text.
%To use type \todo{  insert text here  }.

\newcommand{\todo}[1]{\vspace{5 mm}\par \noindent
\marginpar{\textsc{ToDo}}
\framebox{\begin{minipage}[c]{0.95 \textwidth}
\tt #1 \end{minipage}}\vspace{5 mm}\par}

%%%%%%%%%%%%%%%%%%%%%%%%%%%%%%%%%%%%%%%%%%%%%%%%%%%%%%%%%%%%%%%%%%%%%%%%%%%%%%%

\binoppenalty=10000
\relpenalty=10000

\begin{document}


% Begin the front matter as required by Rackham dissertation guidelines

\initializefrontsections

\pagestyle{title}

\begin{center}
Санкт-Петербургский национальный исследовательский университет \\ информационных технологий, механики и оптики

\vspace{2cm}

Кафедра компьютерных технологий

\vspace{3cm}

{\Large Г. С. Ткаченко}

\vspace{2cm}

\vbox{\LARGE\bfseries
Параллельные алгоритмы поиска \\ кратчайшего пути в графе}

\vspace{4cm}

Бакалаврская работа 

\vspace{1cm}

{\Large Научный руководитель: Г. А. Корнеев}

\vspace{5cm}

Санкт-Петербург\\ 2015
\end{center}

\newpage

\setcounter{page}{3}
\pagestyle{plain}

%\dedicationpage{Put a dedication here}
% Dedication page

%\startacknowledgementspage
% Acknowledgements page
%{Put Acknowledgements here}

% Table of contents, list of figures, etc.
\tableofcontents
%\listoffigures


\def\t#1{\mbox{\texttt{\hbox{#1}}}}
\def\b#1{\textbf{#1}}
\def\tb#1{\t{\b{#1}}}

\def\cln#1{\t{#1}}
\def\pcn#1{\t{#1}}
\newcommand{\p}{\par Здесь будет текст...}

\def\drawfigure#1#2#3{
        \begin{figure}[ht]
        \centerline{ \includegraphics[width=8cm]{img/#1}}
        \caption{#2}
        \label{#3}
        \end{figure}
}
\def\drawfigurex#1#2#3#4{
        \begin{figure}[ht]
        \centerline{ \includegraphics[#4]{img/#1}}
        \caption{#2}
        \label{#3}
        \end{figure}
}

% Chapters
\startthechapters
% -*-coding: utf-8-*-
%$\newcommand{\ncs}{\mbox{N-CROSS} SUM}
%\newcommand{\threecs}{\mbox{3-CROSS} SUM}
%\newcommand{\fourcs}{\mbox{4-CROSS} SUM}
%\newcommand{\sat}{\mbox{1-in-3 SAT}}
%\newcommand{\six}{$(A, B, a_0, b_0, s, V)$}
%\newcommand{\tdld}{\mbox{2-DLD}}
%\newcommand{\false}{\t{КНФЭ}}
%\newcommand{\true}{\t{ХЯРХМЮ}}
%\newcommand{\gadget}{\ttfamily}
%\newcommand{\setA}{$\mathcal A_1$}
%\newcommand{\setB}{$\mathcal A_2$}
%\newcommand{\setC}{$\mathcal A_2'$}

\startprefacepage

Задача определения кратчайшего расстояния до конфигурации отрезков
(dist2segments), являясь частным случаем задачи dist2sites -- задачи о
минимальном расстоянии до произвольного набора линейных данных,
является одной из базовых задач вычислительной геометрии (computational
geometry) \cite{PrSh}. На решении данной задачи базируются решения некоторых задач
об избежании столкновений (collision avoidance problem) \cite{MarNav}, некоторых задачи
из области геоинформационных систем (Geographic information system, GIS) \cite{CGinGIS},
текстурирования рельефа, математического моделирования движения твердых
тел в жидкости и многих других.

Очень часто возникает необходимость в достаточно
больших количествах запросов на поиск ближайших отрезков.
Например, такая необходимость возникает при расчете физики движения судов в морских
тренажерах \cite{MarNav}. Судов может быть достаточно много, объектов, с которыми
необходимо рассчитать взаимодействие тоже достаточно много. Это означает,
что запросы расстояний до ближайших отрезков идут достаточно часто, чтобы это стало проблемой
для ЭВМ, вычислительная мощность которых может быть недостаточно
большой для такого потока данных. Такого рода запросы называются
массовыми запросами, а такая задача -- массовой задачей.

Массовая задача в \cite{PrSh} определена следующим образом: существует
фиксированный набор входных данных $S$. Требуется вычислить массовый
запрос $Q$, то есть ответить на некоторый поставленный вопрос для каждого
запроса из $Q$. Иногда такие задачи решаются в два этапа – предобработка 
(pre-processing) и вычисление запросов на некоторой структуре данных,
формирующейся на этапе предобработки и облегчающей поиск, что позволяет
сократить суммарное время по сравнению с последовательным решением
исходной задачи для каждого запроса.
Логично предположить, что такая структура данных должны быть
подобна хешу (hash) \cite{AHU} или дереву (tree) \cite{QT, SQT, FANN}. В работе \cite{NGRID} была предложена
такая структура данных -- многоуровневая сеть (n-grid). Если потребовать,
чтобы с каждой ячейкой сети ассоциировались только ближайшие к этой
ячейке отрезки, то задача сводится к перебору небольшого числа отрезков. В
данной работе будет рассмотрена древовидная структура -- квадродерево,
позволяющая наиболее эффективно решать массовые геометрические задачи
такого типа, и методы ее построения, а также будет произведено сравнение
этой структуры с многоуровневой сетью и явным построением диаграммы
Вороного.
Вообще, задача определения кратчайшего расстояния до конфигурации
отрезков имеет очевидное решение -- это полный
перебор с отсечением, имеющий линейную сложность ($O(n)$) \cite{DnCG}. Однако, если
множество точек-запросов $Q$ имеет достаточно большую мощность, обычно
рассматривают так называемую диаграмму Вороного для отрезков (segment
Voronoi diagram, SVD) \cite{PrSh, CGAL}. Эта структура данных похожа на диаграмму
Вороного для точек (point Voronoi diagram) \cite{PrSh, CGAL}, однако она не может быть
представлена реберным списком с двойными связями (double-connected edge
list, DCEL, РСДС) \cite{PrSh, CGAL} в силу своей нелинейности. Хотя, конечно, можно
создать приближенный РСДС, сколь угодно точно описывающий диаграмму
Вороного для отрезков.

%-*-coding: utf-8-*-
\chapter{Shortest path from one vertex to every other}


\FloatBarrier
\section{Classic sequential Bellman-Ford}

\FloatBarrier
\begin{algorithm}
\caption{Classic Bellman-Ford}\label{bf_classic_seq}
\begin{algorithmic}[1]
\Procedure{ClassicBellmanFord}{$G,start$}
\State {$dist \gets \left\{ {\infty ... \infty}\right\}$}
\State {$dist[start] \gets 0$}
 
\For{$i = 0$ to $|G.vertices| - 1 $}
	\For{$e \in G.edges $}
		\State $dist[e.to] \gets \min(dist[e.to], dist[e.from] + e.w)$
	\EndFor
\EndFor
\State \textbf{return} $dist$
\EndProcedure
\end{algorithmic}
\end{algorithm}

\FloatBarrier
\section{BFS-like sequential Bellman-Ford}

\FloatBarrier
\begin{algorithm}
\caption{BFS-like BellmanFord}\label{bf_bfs_seq}
\begin{algorithmic}[1]
\Procedure{BFSBellmanFord}{$G,start$}
\State $dist\gets \left\{ {\infty ... \infty}\right\}$
\State $dist[start] \gets 0$
\State $CurrentVertexSet \gets \left\{ {start}\right\}$\Comment{Set of vertices the distance to which has just been updated} 
\State $NextVertexSet \gets \emptyset$ 
\State {$step \gets 0$ }
\While {$step < |G.vertices|$ \algorithmicand \algorithmicnot $ CurrentVertexSet.empty()$}
	\State $step \gets step + 1$
	\State $NextVertexSet.clear()$
	
	\For{$v \in CurrentVertexSet$}
		\For{$e \in G.edgesFrom[v] $} \Comment{Outgoing edges of current vertex} 
			\If {$dist[e.to] < dist[e.from] + e.w$} 
				\State $dist[e.to] \gets dist[e.from] + e.w$
				\State $NextVertexSet.insert(e.to)$								
			\EndIf
		\EndFor
	\EndFor
	
	\State $CurrentVertexSet \gets NextVertexSet$	
\EndWhile
\State \textbf{return} $dist$

\EndProcedure
\end{algorithmic}
\end{algorithm}


\FloatBarrier
\subsection{Parallel Bellman-Ford by edges of current vertex}
Idea : use parallel min reduce for incoming edges of current vertex   


\FloatBarrier
\begin{algorithm}
\caption{Parallel Bellman-Ford by edges of current vertex}\label{bf_classic_par1}
\begin{algorithmic}[1]
\Procedure{BellmanFordPar1}{$G,start$}
\State $dist\gets \left\{ {\infty ... \infty}\right\}$
\State $dist[start] \gets 0$
 
\For{$i = 0$ to $|G.vertices| - 1 $}
	\State {$changed \gets $ \algorithmicfalse}
	\For{$v \in G.vertices $}
		\algrenewcommand\algorithmicfor{\textbf{parfor}}
		\State {$minDist \gets $ min reduce by G.inEdges[v]} 
		
		\If {$dist[v] > minDist$} 
			\State $dist[v] \gets minDist$
			\State {$changed \gets $ \algorithmictrue}						
		\EndIf
		\algrenewcommand\algorithmicfor{\textbf{for}}

	\EndFor
	\If {\algorithmicnot $changed$} 
		\State $break$
	\EndIf
\EndFor
\State \textbf{return} $dist$
\EndProcedure
\end{algorithmic}
\end{algorithm}

\FloatBarrier
\subsection{Parallel Bellman-Ford by all edges using prefixsum}

Idea : use precalculated prefixsum to divide current vertex set for 2 sets, which will be handled by different threads


\FloatBarrier
\begin{algorithm}
\caption{Parallel Bellman-Ford by all edges using prefixsum}\label{bf_classic_par2}
\begin{algorithmic}[1]
\Procedure{BellmanFordPar2}{$G,start$}
\State $dist\gets \left\{ {\infty ... \infty}\right\}$
\State $dist[start] \gets 0$
\State {$prefsum \gets $ prefix sum by vertices incoming degree} 
\State {$planMap \gets $ \Call {BuildPlan}{$prefsum$, 0, |$G.vertices$|}} 

\For{$i = 0$ to $|G.vertices| $}	
	\If {\algorithmicnot \Call {ProcessLayer}{$G, planMap, prefsum, 0, |G.vertices|$}} 
		\State \textbf{break}
	\EndIf
		
\EndFor
\State \textbf{return} $dist$
\EndProcedure

\State 
\Procedure{BuildPlan}{$prefsum, startV, endV$}  \Comment{This function returns a structure which tells us the middle of the segments (by index of start and end vertex)}
\State {$resultMap \gets $ empty map}
\State $edgesNumber \gets prefsum[endV] - prefsum[startV]$
\If {$edgesNumber < threshold$} 
	\State $midV \gets $ binary search by edges number
	\State $resultMap[startV][endV] \gets midV$ 
	\State {$resultMap.addAll($ \Call {BuildPlan}{$prefsum, startV, midV$})} 
	\State {$resultMap.addAll($ \Call {BuildPlan}{$prefsum, midV, endV$})} 
\EndIf

\State \textbf{return} $resultMap$
\EndProcedure

\State 
\Procedure{ProcessLayer}{$G, planMap, prefsum, startV, endV$}  
\State $edgesNumber \gets prefsum[endV] - prefsum[startV]$
\If {$edgesNumber < threshold$} 
	\State process vertices sequentally 	
\Else	
	\State $midV \gets planMap[startV][endV]$ 
	\State {\Call {ProcessLayer}{$G, planMap, prefsum, startV, midV$}}
	\State {\Call {ProcessLayer}{$G, planMap, prefsum, midV, endV$}}
\EndIf

\EndProcedure

\end{algorithmic}
\end{algorithm}

\FloatBarrier
\subsection{Parallel BFS-like Bellman-Ford}
Idea : use your PBFS to handle vertex distances

\FloatBarrier
\begin{algorithm}
\caption{Parallel BFS-like Bellman-Ford}\label{bf_bfs_par}
\begin{algorithmic}[1]
\Procedure{BellmanFordPar3}{$G,start$}
\State $dist\gets \left\{ {\infty ... \infty}\right\}$
\State $dist[start] \gets 0$
\State {$Frontier \gets \left\{ {G.edgesFrom(start)}\right\}$}
\For{$i = 0$ to $|G.vertices| $}	
	\State {$Frontier \gets $  {\Call {HandleFrontier}{$Frontier$}}} \Comment{relax edges from Frontier and build a new one} 
	
	\If { $Frontier.empty()$} 
		\State $break$						
	\EndIf
		
		
\EndFor
\State \textbf{return} $dist$
\EndProcedure
\State
\Procedure{HandleFrontier}{$Frontier$}
\State recursively divide current frontier, atomically relax edges in frontier and building a new one

\State \textbf{return} $NewFrontier$
\EndProcedure

\end{algorithmic}
\end{algorithm}


\FloatBarrier
\subsection{Algo comparison}
At the first sight it may seems that Algorithm 3 has only disadvantages. The main problem is that it has bad parallelisation ability compared to other two algorithms. But sometimes it works better even on 40-core machine because of one useful property. Let's consider a graph where all the edges have the from $i \rightarrow j$ where $i < j$. Once the iteration number $I$ has passed all the vertices till $I$ have correct target distance. It' easy to prove using mathematic induction. So it means that we have to perform only 2(!!!) loops in that case. One for calculating distance and one for understanding that nothing will change anymore. And let's assume that we have a dense graph. In that circumstances Algorithm 4 will suffer from great parallelism (the number of iterations of the main loop will increase significantly) and Algorithm 5 will have to handle the large set of vertices (because graph is dense) during the iterations. 

But anyway it's easy to see that in the most cases Algorithm 4 and Algorithm 5 will beat Algorithm 3. Let's compare them. As I said before Algorithm 5 will be not good enough when we're considering dense graphs, because of big size of queue. So the main recommendation of when to use these approaches is to realise if the graph is dense or sparse. In the first case you have to use Algorithm 4, otherwise Algorithm 5.


\FloatBarrier
\subsection{Testing}

Now we'll prove our assumptions on practice. I've implemented all the algorithms and compared them. Description of input graphs is presented in the Table 1.1. The results are presented in the Table 1.2
\FloatBarrier

\begin{table}[H]
\centering

\begin{tabular}{c|c|c}  
Name & Vertices & Edges\\
\hline\hline
CompleteTS sign(-) & 7071 & 24995985 \\  
Complete sign & 3162 & 9995082  \\  
BalancedTree fraction & 8388607 & 8388608 \\  
SquareGrid sign & 2499561 & 4999122  \\  
RandomSparse fraction(0.5) sign & 2500000 & 25000000  \\  
RandomSparse fraction(0.96) sign(+) & 2500000 & 25000000  \\  
RandomDense fraction(0.5) sign & 5000 & 25000000  \\  
RandomDense fraction(0.96) sign(+) & 5000 & 25000000  \\  
\hline
\multicolumn{3}{l}{\footnotesize \textit{sign} - sign of weights on edges }\\
\multicolumn{3}{l}{\footnotesize \textit{fraction} - fraction of lexicographically sorted edges (edges of type X -> X+i) }\\
\multicolumn{3}{l}{\footnotesize \textit{TS} - exists only Lexicographically Sorted edges (fraction = 1) }\\
\multicolumn{3}{l}{\footnotesize }\\
\multicolumn{3}{l}{\footnotesize  expression "\textit{RandomDense fraction(0.5) sign}" means Random Dense }\\
\multicolumn{3}{l}{\footnotesize 	graph with specified fraction and any sign of weights}\\
\end{tabular}

\caption{Input graph description}
\label{bf_algo_comparison}
\end{table}
\FloatBarrier

\begin{table}[H]
\centering

\begin{tabular}{l|ccc|cc|cc|ccc|ccc}  
Algo №& \multicolumn{3}{c}{Complete} & \multicolumn{2}{c}{BalancedTree} & \multicolumn{2}{c}{SquareGrid} & \multicolumn{3}{c}{RandomSparse} & \multicolumn{3}{c}{RandomDense}\\
& TS- & + & - & 0.5 & 1 & + & +- & 0.5+  & 0.5- & 0.96+ & 0.5+ & 0.5- & 0.96+\\
\hline\hline
3 & 2.43 & 4.65 & nc & 116.31 & 9.04 & 5.49 & 13.40 & nc & nc & 24.35 & nc & nc & 5.01 \\  
4 & 5.17 & 0.18 & 10.84 & 3.59 & 3.08 & 5.92 & 7.10 & 2.77 & 14.68 & 2.42 & 0.48  & 6.38  & 0.46 \\
5 & 44.63 & 0.37 & 23.55 & 0.44 & 0.31 & 4.42 & 0.58 & 0.98 & 22.59 & 0.76  & 0.60  & 10.25 & 0.71 \\
\hline
\end{tabular}

\caption{Bellman-Ford algorithms comparison}
\label{graph_description}
\end{table}

\FloatBarrier
\section{Conclusion}

You can easily find out from tables that our assumptions were correct. Algorithm 3 works good for dense graph with very high fraction (almost 1), Algorithm 4 is good for dense graphs and for graphs with negative edges, Algorithm 5 is good for sparse graph with positive edges. 

\FloatBarrier

%-*-coding: utf-8-*-
\chapter{Решение массовой задачи о ближайшем отрезке с использованием квадродерева}
\section{Квадродерево}
Квадродерево -- поисковая структура данных, которая хранит в себе
подразбиение плоскости и позволяет быстро производить локализацию точек-
запросов. Узел (ячейка) квадродерева представляет собой прямоугольник, для
которого определена некоторая мера его насыщенности $p(C)$. Если узел
насыщен ($p(C) > T$, где $T$ -- предельное насыщение), то происходит его
разбиение на четыре одинаковых дочерних узла (делением пополам по
вертикали и по горизонтали). Таким образом, в каждый момент времени у узла
или нет детей или их четыре. Разбиение происходит до тех пор, пока все узлы
не перестанут быть насыщенными или не будет достигнута максимальная
глубина подразбиения. Ограничение глубины подразбиения играет важную
роль в виду того, что не всегда получается сделать узел ненасыщенным за
конечное (или разумное) количество разбиений, при описании применения
квадродерева в предложенном алгоритме, будет дано более точное обоснование
необходимости ограничения.

Квадродеревья и их модификации очень часто применяют для решения
задач примерного поиска ближайшего соседа (Approximate Nearest Neighbor
Search) для точек \cite{FANN} (рис. \ref{ann}). В качестве меры насыщения в этой задаче часто
выбирают количество точек, среди которых производится поиск, попавших в
ячейку. В насыщенность обычно ограничивают одной точкой в одной ячейке.
Максимальная глубина древа для $n$ точек может составлять $n$, в результате чего
время локализации может составлять $O(n)$. Для борьбы с этим была разработана
структура данных Skip-Quadtree \cite{SQT}, которая позволяет производить локализацию за $O(\log n)$.

\drawfigure{ann}{Квадродерево}{ann}

\section{Нижняя огибающая}
Неформально нижняя огибающая (lower envelope) множества объектов на плоскости –
множество точек этих объектов, видимые наблюдателем, расположенным в
точке $(0, -\infty)$. Формально же это граф, представляющий из себя поточечный
минимум кусочно-заданных функций \cite{LENV} (рис. \ref{lenv}).
Также наряду с нижней огибающей часто рассматривается минимизационная диаграмма
(minimization diagram), которая представляет собой проекцию нижней огибающей на
горизонтальную ось (рис. \ref{mdiag}). 
\drawfigure{lenv}{Нижняя огибающая}{lenv}
\drawfigure{mdiag}{Минимизационная диаграмма}{mdiag}

\section{Алгоритм}
\subsection{Идея алгоритма}
Основной идеей всех алгоритмов поиска ближайших сайтов (sites),
основанных на подразбиении пространства, является растеризация (с явным
построением или без него) диаграммы Вороного в этом подразбиении. После
этого в ячейках подразбиения оказывается информация, обо всех ближайших
сайтов для всех точек этой ячейки. Поиск ближайшего сайта происходит
путем локализации в этом подразбиении и последующим перебором всех
сайтов, ближайших к найденной ячейке.

В виду нетривиальности задачи поиска всех сайтов ближайших к ячейке,
во многих алгоритмах переходят к примерному решению задачи поиска
ближайшего отрезка \cite{NGRID}, производя поиск сайтов, ближайших к каким-то
точкам ячейки. Точки обычно выбираются таким образом, чтобы обеспечить
заданную точность, но в некоторых случаях даже не идет речи о точности \cite{AVOR}.
Для некоторых случаев погрешность допустима, но робастность (robustness)
является важной характеристикой алгоритмов вычислительной геометрии \cite{ROBUS}.
Предложенный алгоритм позволяет произвести точный поиск ближайших
отрезков для ячеек, при условии, что можно явно (хотя бы кусочно) задать
расстояние от границ ячеек до отрезков в виде полинома.

\subsection{Работа алгоритма}

Для отрезков строится ограничивающий прямоугольник (bounding box), этот прямоугольник будет первым
уровнем квадродерева. В качестве меры насыщенности узла берется количество
ближайших отрезков к данной ячейке. Для первого узла ближайшими будут все
отрезки, так как они все лежат внутри. Далее происходит рекурсивное
подразбиение узлов. Ближайшими к дочернему узлу будут отрезки ближайшие
к его родителю, так как дочерний узел геометрически лежит внутри
родительского. Необходимо произвести фильтрацию лишних отрезков.

{\prop\label{cl_segs}
Ближайшие отрезки для точек ячейки – это отрезки ближайшие к ее границе и отрезки пересекающие ячейку}
\proof Обозначим: $S$ -- множество отрезков ближайших к ячейке, $S_b$ -- ближайших к границе, $S_i$ --
пересекающих ячейку.
\begin{itemize}
\item $S_b \cup S_i \subset S$ \\
$S_b \subset S$ -- очевидно, так как граница ячейки -- это ее подмножество.\\
Для любой точки на пересечении отрезка и ячейки этот отрезок будет
ближайшим, значит $S_i \subset S$
\item $S_b \cup S_i \supset S$ \\
Предположим, что это не так.\\Пусть $s$ -- отрезок, не пересекающий
ячейку, и он не является ближайшим ни к одной точке на границе. Пусть
он ближайший для точки $P$ ячейки, а $Q$ -- точка $s$, ближайшая к $P$.
Построим отрезок $PQ$, так точка $P$ вне ячейки, а точка $Q$ внутри, то $PQ$
пересечет границу ячейки, допустим в точке $F$. Рассмотрим отрезок $s'$,
ближайший к $F$. Пусть точка $E$ -- ближайшая точка на нем к $F$.
Так как $s'$ ближайший к $F$, то $|FE| < |FQ|$, по неравенству треугольника $|PF|
+ |FE| < |PE|$. Подставив первое неравенство во второе, мы получим, что
отрезок $s'$ ближе к $P$ чем $s$ (рис. \ref{contrex}). Противоречие.
\end{itemize}
\drawfigure{contrex}{Противоречие}{contrex}

Это простое утверждение показывает, что для фильтрации нам
необходимо взять из отрезков только те, которые являются ближайшими для
границы ячейки, и те, которые ее пересекают.
Для поиска отрезков ближайших к границе ячейки для каждой стороны
прямоугольника строится нижняя огибающая функций кратчайшего расстояния от
стороны до отрезков, которые фильтруются (рис. \ref{le_dist}).

Функция кратчайшего расстояния до отрезка состоит из трех частей: двух функции расстояния от
стороны до концов отрезка, и функции расстояния от стороны до прямой,
содержащей отрезок этот, заданной на ограниченном промежутке. В результате
из нижней огибающей можно выделить информацию об отрезках ближайших к
сторонам ячейки. Также эта фильтрация оставляет отрезки, пересекающие
границу. Значит, к полученным отрезкам остается только добавить отрезки
лежащие внутри ячейки. Для проверки этого условия достаточно проверить
принадлежность одной из точек отрезка ячейке.

\drawfigurex{le_dist}{Нижняя огибающа функций кратчайших расстояний}{le_dist}{width=6cm}

Подразбиение будет происходить до тех пор, пока все ячейки не
перестанут быть насыщенными, или пока не будет достигнута максимальная
глубина подразбиения. В данной задаче очень важно ограничить глубину
подразбиения, так как в вырожденных случаях (degenerate cases) некоторые
ячейки поразбить не получится. Вырожденным случаем для диаграммы
Вороного является наличие четырех и более сайтов равноудаленных от одной
точки. В таком случае в этой точке получается вершина диаграммы Вороного,
граничащая с ячейками соответствующих сайтов. В результате наличия
большого числа сайтов расположенных таким образом (пусть их $n$), ячейка
квадродерева, содержащая эту точку, будет ближайшей как минимум к $n$
сайтам. Разбив такую ячейку мы все равно получим одну ячейку содержащую
эту вершину диаграммы Вороного. Поэтому имеет смысл ограничивать
глубину разбиения.
Итак, в результате получается квадродерево, в листьях которого лежит
информация о ближайших к ним отрезках. Поиск ближайшего отрезка по такой
структуре данных осуществляется в два этапа. Сначала происходит
локализация точки-запроса в квадродереве. Затем перебираются все отрезки,
ближайшие к найденной ячейке, и среди них выбирается ближайший.

\subsection{Оценка числа перебираемых отрезков}
Скорость обработки запроса очень сильно зависит от числа отрезков в ячейке.
Рассмотрим равномерное распределение точек-запросов на прямоугольнике,
задаваемом верхним уровнем квадродерева. Пусть $X$ -- случайный запрос, 
$n(X)$ - число перебираемых отрезков, при запросе $X$, $C$ -- прямоугольник, 
покрываемый верхним уровнем квадродерева, $L$ -- множество листьев квадродерева.
\begin{equation}
E\{n(X)\} = \int\limits_Cn(X)dp(X) =  \sum\limits_{l \in L}n_lp_l
\end{equation}
Так как распределение равномерное, то $p_l = \frac{S_l}{S}$, где 
$S_l$ -- площадь листа $l$, $S$ - площадь покрываемая квадродеревом.
В итоге получаем простую формулу.
\begin{equation}
E\{n(X)\} = \frac{1}{S}\sum\limits_{l \in L}n_lS_l
\end{equation}

Попытаемся оценить эту величину сверху. Пусть губина дерева ограничена числом $d$, 
а $c$ -- насыщенность узла квадродерева, после которой он разбивается.
Заметим, что листья бывают двух видов: насыщенные и ненасыщенные. Насыщенние листья -- это
листья, которые находятся на уровне $d$ и все еще содержат больше, чем $c$ отрезков.
Обозначим множество ненасыщенных листьев $G$ (good), множество насыщенных листьев $B$ (bad).
\begin{equation}
E\{n(X)\} = \frac{1}{S}\sum\limits_{l \in L}n_lS_l = 
\label{expectation}
\frac{1}{S}\sum\limits_{l \in G}n_lS_l 
+
\frac{1}{S}\sum\limits_{l \in B}n_lS_l
\end{equation}
Оценим первую сумму.
\begin{equation}
\frac{1}{S}\sum\limits_{l \in G}n_lS_l \le \frac{c}{S}\sum\limits_{l \in G}S_l \le \frac{c}{S}S = c
\label{sum1}
\end{equation}
Оценим вторую сумму. Так как листья из $B$ находятся на уровне $d$, то $S_l = \frac{S}{4^d}$.
В каждом таком листе не более $n$ отрезков.
\begin{equation}
\frac{1}{S}\sum\limits_{l \in B}n_lS_l \le \frac{Sn}{4^dS}|B| = \frac{n|B|}{4^d}
\label{sum2}
\end{equation}
Осталось оценить мощность множества $B$. Ячейка попадает в множество $B$, если ее пересекает
большое число локусов диаграммы Вороного исходного множества отрезков.
Это возможно в случае попадания туда вершины диаграммы Вороного большой степени (рис. \ref{big_deg})
или пересечения большим числом узких локусов (рис. \ref{thin_loc}). Вершин в диаграмме Вороного
$O(n)$, локусов тоже $O(n)$. 

Рассмотрим подразбиение прямоугольника, покрываемого верхним уровнем 
диаграммы Вороного, на $4^d$ ячеек ($2^d$ по вертикали и горизонтали).
Границы локусов представляют из себя отрезки прямых и парабол, следовательно
узкие локусы растеризуются в этой сетке как отрзки или параболы, а не как площадные объекты, 
так они уже ячеек сетки. При растеризации в сетку параболы и отрезки пересекают $O(n)$ ячеек (сетка $n \times n$).
Следовательно узкий локус пересекает $O(2^d)$ ячеек. Можно оценить $|B|$.
\begin{equation}
|B| = O(n2^d)
\label{bad_segs}
\end{equation}

Сводя все воедино, получаем верхнюю оценку мат ожидания.
\begin{equation}
E\{n(X)\} = c + \frac{nO(n2^d)}{4^d} = c + \frac{O(n^2)}{2^d}
\end{equation}

Отсюда видно, что при $d = O(\log n)$ $E\{n(X)\} = O(1)$.

\subsection{Анализ полученных результатов}
Если произвести грубую оценку времени построения квадродерева для
этой задачи, то получается, что оно ограничено только максимальной глубиной
подразбиения. Это так, но на практике построение происходит достаточно
быстро, в виду того, что сильное подразбиение испытывают в основном
области, содержащие вершины SVD. Максимально возможное число ячеек
будет $4^d$, где $d$ -- максимальная глубина подразбиения.

Скорость поиска ближайшего отрезка складывается из скорости
локализации и скорости поиска ближайшего отрезка среди ближайших к
ячейке. Тогда как первая величина ограничена сверху $d$, вторая ограничивается
только количеством отрезков (достаточно вспомнить вырожденный случай).
Ввиду того что $d$ обычно не очень велико и локализация в дереве -- не
ресурсоемкая операция, очень показательной величиной оказывается среднее
число отрезков перебираемых при запросе. Эта величина является суммой по
всем листовым ячейкам дерева количества ближайших отрезков, помноженных
на долю площади, покрытой этой ячейкой. На практике такая структура
оказывается очень эффективной, она позволяет давать точные ответы на
запросы о ближайших отрезках в среднем за $O(1)$, а среднее число
перебираемых отрезков держится в промежутке $0,5k ÷ k$, где $k$ -- предельная
насыщенность ячейки.

В следующей главе будет приведено сравнение
квадродерева с n-grid \cite{NGRID} и SVD \cite{CGAL}.
Максимальное количество занимаемой памяти можно грубо оценить, как
O($4^{d}n$), что на практике не наблюдается.

\startconclusionpage

Как было показано в данной работе, задача определения кратчайшего
расстояния до произвольной конфигурации отрезков на плоскости представляет
достаточный практический интерес. Были подробно рассмотрены применяемые
на практике подходы. Предложенный подход не только позволяет искать
ближайшие отрезки быстрее общепринятых методов, но еще и гораздо проще в
реализации. Из недостатков данного подхода стоит отметить лишь немного большее, по
сравнению с другими рассмотренными методами, потребление памяти. Но для
многих задач данные затраты оказываются вполне приемлемыми.

В данный момент большой практический интерес представляет ускорение
построения нижних огибающих за счет отказа от точной арифметики в пользу более
быстрых альтернативных техник, таких как, например, adaptive precision
arithmetic \cite{APREC}. Это позволит сократить время на построение, как минимум в
десять раз. Также интересным было бы разработать технику отсечения
заведомо далеких отрезков перед построением нижних огибающих. Что более
важно, реализованный алгоритм очень просто поддается распараллеливанию,
ввиду полной независимости обработки поддеревьев.

Также хотелось бы отметить, что хоть данный метод применялся для
отрезков, он легко обобщается и для других классов объектов. К тому же
данный подход применим и для других классов задач, таких как поиск наиболее
удаленного отрезка (furthest segment problem) и поиск $k$ ближайших соседей
(k-Nearest Neighbor Search). Также данный метод не требует построения
пересечений отрезков для своей работы, в отличии от других рассмотренных
методов.

\FloatBarrier


%\startappendices
%\label{appendix}
%\input{appendix}

\bibliographystyle{sty/utf8gost705u}
\bibliography{thesis}

\end{document}
	