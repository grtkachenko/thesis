% -*-coding: utf-8-*-
\startprefacepage

Алгоритмы поиска кратчайших путей на графах нашли свое применение в различных областях и сферах деятельности человека. Такие алгоритмы используются в картографических сервисах, при построении пути GPS-навигатора, для представления и анализа дорожной сети и во многих других областях.
 
При этом в настоящее время существует большое число алгоритмов и подходов, которые решают эту задачу. И все алгоритмы можно логически разделить на два класса - алгоритмы поиска кратчайшего расстояния от одной вершины до всех остальных и алгоритмы поиска кратчайших расстоянии между каждой парой вершин. Из первого класса самыми яркими представителями являются различные модификации алгоритмов Дейкстры и Беллмана-Форда. Для решения задач второго класса часто используются алгоритмы Флойда-Уоршелла и алгоритм Джонсона. 

Однако с ростом многопроцессорных архитектур встала задача по возможности запускать эти алгоритмы на нескольких вычислительных ядрах. Именно такие параллельные версии алгоритмов будут освещены в моей работе. В первом части представлены параллельные модификации алгоритма Беллмана-Форда. Во второй части работы представлен алгоритм по поиску кратчайших расстояний между каждой парой вершин в общем случае и эффективная модификация для поиска расстояний в социальных графах.

\FloatBarrier
