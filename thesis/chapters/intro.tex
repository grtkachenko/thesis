% -*-coding: utf-8-*-
\startprefacepage

Алгоритмы поиска кратчайших путей на графах нашли свое применение в различных областях и сферах деятельности человека. Такие алгоритмы используются в картографических сервисах, при построении пути GPS-навигатора, для представления и анализа дорожной сети и во многих других областях.
 
При этом в настоящее время существует большое число алгоритмов и подходов, которые решают эту задачу. При этом в большинстве своем алгоритмы можно логически разделить на два класса --- алгоритмы поиска кратчайшего расстояния от одной вершины до всех остальных и алгоритмы поиска кратчайших расстоянии между каждой парой вершин. Из первого класса самыми яркими представителями являются различные модификации алгоритмов Дейкстры и Беллмана-Форда. Для решения задач второго класса часто используются алгоритмы Флойда-Уоршелла и алгоритм Джонсона. 

С ростом многопроцессорных архитектур мы получили мощный инструмент для более эффективного расчета искомых расстояний --- мы получили возможность запускать эти алгоритмы на нескольких вычислительных ядрах. При этом в контексте с параллельными алгоритмами на графах встал вопрос об эффективном использовании ресурсов системы. Эта задача, однако, не имеет такого высокого разнообразие решений, как в случае однопоточного алгоритма. Именно над этой проблемой я работал --- в статье освещены различные версии параллельных алгоритмов поиска кратчайшего пути в графах. В первой главе представлены параллельные модификации алгоритма Беллмана-Форда. Во второй главе работы представлен параллельный алгоритм по поиску кратчайших расстояний между каждой парой вершин в общем случае и эффективная модификация для поиска расстояний в социальных графах. При этом во многих алгоритмах будет использоваться современные и высокопроизводительные структуры данных и подходы для параллельной обработки ребер графа.

Все представленные алгоритмы были реализованы на основе библиотеки для параллельных вычислений и протестированы на различных графовых структурах. При этом описанные подходы продемонстрировали высокую скорость работы на реальных графах, что подтверждает их применимость в реальной жизни.
\FloatBarrier
