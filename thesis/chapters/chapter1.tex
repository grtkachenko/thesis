%-*-coding: utf-8-*-
\chapter{Решение задачи поиска кратчайшего расстояния от фиксированной вершины до всех остальных}
\label{chapSVD}

В данной главе описаны алгоритмы по решению классической задаче на графах - поиску кратчайших расстояний от одной вершины до всех остальных. В первой части главы представлен краткий обзор предметной области. Во второй параллельные модификации алгоритма Беллмана-Форда. 

\FloatBarrier
\section{Обзор существующих решений}
\subsection{Алгоритм Дейкстры}

Одним из наиболее заметных алгоритмов для решения данной задачи является алгоритм Дейкстры. Придуманный еще в 1959 году Эдсгером Вибе Дейкстрой он сохраняет свою актуальность и по сей день. Основная идея состоит в последовательном пополнении множества вершин, расстояние для которых уже корректно посчитано. При этом на каждом шаге выбирается вершина, которая находится ближе остальных к уже посчитанному множеству. 

Существует множество модификации алгоритма основанных на различных структурах данных для выбора вершины с минимальный расстоянием на каждом из шагов алгоритма. В зависимости от этого алгоритм может работать $O(V^2+E)$, $O(E \log V)$ или $O(V \log V+E)$. 

Основная проблема алгоритма состоит в том, что он работает только на графах с неотрицательным весом ребер. С этой проблемой справляется алгоритм Беллмана-Форда. 


\FloatBarrier
\subsection{Алгоритм Беллмана-Форда}

Классический алгоритм Беллмана-Форда работает на графах с произвольным весом ребер, однако имеет заметно худшую асимптотику по сравнению с алгоритмом Дейсктры - $O(VE)$. 

Основная идея алгоритма основана на идее динамического программирования. После k итерации алгоритма утверждается, что будут корректно посчитаны и обработаны значения веса путей длиной не более K. И после V итерации расстояние до каждой из вершин посчитано корректно. Ниже приведен каноничный псевдокод алгоритма. 

\FloatBarrier
\begin{algorithm}
\caption{Классический алгоритм Беллмана-Форда}\label{bf_classic_seq}
\begin{algorithmic}[1]
\Procedure{ClassicBellmanFord}{$G,start$}
\State {$dist \gets \left\{ {\infty ... \infty}\right\}$}
\State {$dist[start] \gets 0$}
 
\For{$i = 0$ to $|G.vertices| - 1 $}
	\For{$e \in G.edges $}
		\State $dist[e.to] \gets \min(dist[e.to], dist[e.from] + e.w)$
	\EndFor
\EndFor
\State \textbf{return} $dist$
\EndProcedure
\end{algorithmic}
\end{algorithm}

\FloatBarrier
Кроме того существует интересная модификация алгоритма, которая поддерживает на каждой итерации набор вершин, расстояние до которых изменились на предыдущем шаге алгоритма. Из очевидных соображений мы имеем право рассматривать только эти и никакие другие вершины. Этот алгоритм на практике зачастую работает заметно быстрее чем классическая версия в некоторых случаях, но об этом подробно будет описано позднее. Будем называть эту версию BFS-подобный Беллман-Форд. Ниже приведен псевдокод алгоритма 

\FloatBarrier
\begin{algorithm}
\caption{BFS-подобный Беллман-Форд}\label{bf_bfs_seq}
\begin{algorithmic}[1]
\Procedure{BFSBellmanFord}{$G,start$}
\State $dist\gets \left\{ {\infty ... \infty}\right\}$
\State $dist[start] \gets 0$
\State $CurrentVertexSet \gets \left\{ {start}\right\}$\Comment{структура данных для хранения набора вершин} 
\State $NextVertexSet \gets \emptyset$ 
\State {$step \gets 0$ }
\While {$step < |G.vertices|$ \algorithmicand \algorithmicnot $ CurrentVertexSet.empty()$}
	\State $step \gets step + 1$
	\State $NextVertexSet.clear()$
	
	\For{$v \in CurrentVertexSet$}
		\For{$e \in G.edgesFrom[v] $} \Comment{исходящие ребра из текущей вершины} 
			\If {$dist[e.to] > dist[e.from] + e.w$} 
				\State $dist[e.to] \gets dist[e.from] + e.w$
				\State $NextVertexSet.insert(e.to)$								
			\EndIf
		\EndFor
	\EndFor
	
	\State $CurrentVertexSet \gets NextVertexSet$	
\EndWhile
\State \textbf{return} $dist$

\EndProcedure
\end{algorithmic}
\end{algorithm}

\FloatBarrier
\subsection{Другие алгоритмы}

Также известны специализированные алгоритмы, такие как алгоритм A* и D*, которые оперирует большими специализированными графами и используют ряд эвристик для поиска расстояний. При этом в контексте наших исследований они затронуты не будут.

\FloatBarrier
\section{Параллельный алгоритм Беллмана-Форда}

В предыдущей главе были рассмотрены классические алгоритмы поиска кратчайших путей в графе. В этой главе будет рассмотрен алгоритм Беллмана-Форда в контексте параллельных вычислений. Кроме того будем использовать в каждом из алгоритмов идею ранней остановки - если на текущем шаге ни одно из значений массива расстояний не изменилось, то имеем право выйти из основного цикла. В последующих главах будет представлено несколько версий алгоритма, а также их последующее сравнение и рекомендации по использованию.  

\FloatBarrier
\subsection{Параллелизация по ребрам вершины}
Первая версия алгоритма основана на параллельной обработке всех ребер, исходящих из текущей вершины. Псевдокод, который практически не отличается от классической версии, приведен ниже. 


\FloatBarrier
\begin{algorithm}
\caption{Параллельный Беллман-Форд по ребрам вершины}\label{bf_classic_par1}
\begin{algorithmic}[1]
\Procedure{BellmanFordPar1}{$G,start$}
\State $dist\gets \left\{ {\infty ... \infty}\right\}$
\State $dist[start] \gets 0$
 
\For{$i = 0$ to $|G.vertices| - 1 $}
	\State {$changed \gets $ \algorithmicfalse}
	\For{$v \in G.vertices $}
		\algrenewcommand\algorithmicfor{\textbf{parfor}}
		\For{$e \in G.fromEdges[v] $} 
			\If {$dist[e.to] < dist[e.from] + e.w$} 
				\State $dist[e.to] \gets dist[e.from] + e.w$
				\State {$changed \gets $ \algorithmictrue}						
			\EndIf
		\EndFor	
		\algrenewcommand\algorithmicfor{\textbf{for}}

	\EndFor
	\If {\algorithmicnot $changed$} 
		\State $break$
	\EndIf

\EndFor
\State \textbf{return} $dist$
\EndProcedure
\end{algorithmic}
\end{algorithm}

\FloatBarrier
\subsection{Параллелизация по всем ребрам}
Идея второго алгоритма состоит в разбиений всего набора вершин на некоторые подмножества, каждое из которых будет обрабатываться отдельным процессором. При этом для каждой вершины будем рассматривать набор ребер, входящих в нее. Это необходимо для того, чтобы обновление фиксированной ячейки в массиве расстояний происходило только одним потоком. 

\FloatBarrier
\begin{algorithm}
\caption{Параллельный Беллман-Форд по всем ребрам}\label{bf_classic_par2}
\begin{algorithmic}[1]
\Procedure{BellmanFordPar2}{$G,start$}
\State $dist\gets \left\{ {\infty ... \infty}\right\}$
\State $dist[start] \gets 0$
\State {$prefsum \gets $ prefix sum by vertices incoming degree} 
\State {$planMap \gets $ \Call {BuildPlan}{$prefsum$, 0, |$G.vertices$|}} 

\For{$i = 0$ to $|G.vertices| $}	
	\If {\algorithmicnot \Call {ProcessLayer}{$G, planMap, prefsum, 0, |G.vertices|$}} 
		\State \textbf{break}
	\EndIf
		
\EndFor
\State \textbf{return} $dist$
\EndProcedure

\State 
\Procedure{BuildPlan}{$prefsum, startV, endV$}  \Comment{Функция возвращают структуру, которая по отрезку возвращает его середину}
\State {$resultMap \gets $ empty map}
\State $edgesNumber \gets prefsum[endV] - prefsum[startV]$
\If {$edgesNumber < threshold$} 
	\State $midV \gets $ binary search by edges number
	\State $resultMap[startV][endV] \gets midV$ 
	\State {$resultMap.addAll($ \Call {BuildPlan}{$prefsum, startV, midV$})} 
	\State {$resultMap.addAll($ \Call {BuildPlan}{$prefsum, midV, endV$})} 
\EndIf

\State \textbf{return} $resultMap$
\EndProcedure

\State 
\Procedure{ProcessLayer}{$G, planMap, prefsum, startV, endV$}  
\State $edgesNumber \gets prefsum[endV] - prefsum[startV]$
\If {$edgesNumber < threshold$} 
	\State process vertices sequentally 	
\Else	
	\State $midV \gets planMap[startV][endV]$ 
	\State {\Call {ProcessLayer}{$G, planMap, prefsum, startV, midV$}}
	\State {\Call {ProcessLayer}{$G, planMap, prefsum, midV, endV$}}
\EndIf

\EndProcedure

\end{algorithmic}
\end{algorithm}

\FloatBarrier
\subsection{Параллелизация BFS - версии}
Предыдущие две версии были основаны на параллелизации классической версии Беллмана-Форда. В основе следующего алгоритма лежит идея обхода в ширину (Алгоритм 2). В качестве основы для параллельной версии такого алгоритма был взят параллельный обход в ширину, предложенный Умутом Акаром и Майком Рэйни. Подробнее о внутреннем устройстве их подхода речь зайдет во 2 части работы в контексте решения задачи поиска расстояний между каждой парой вершин. Пока лишь приведем псевдокод 

\FloatBarrier
\begin{algorithm}
\caption{Параллельный BFS-подобный Беллман-Форд}\label{bf_bfs_par}
\begin{algorithmic}[1]
\Procedure{BellmanFordPar3}{$G,start$}
\State $dist\gets \left\{ {\infty ... \infty}\right\}$
\State $dist[start] \gets 0$
\State {$Frontier \gets \left\{ {G.edgesFrom(start)}\right\}$}\Comment{структура данных для хранения исходящих ребер текущего множества } 

\For{$i = 0$ to $|G.vertices| $}	
	\State {$Frontier \gets $  {\Call {HandleFrontier}{$Frontier$}}} \Comment{релаксируем ребра из фронтира и строим новый} 
	
	\If { $Frontier.empty()$} 
		\State $break$						
	\EndIf
		
		
\EndFor
\State \textbf{return} $dist$
\EndProcedure
\State
\Procedure{HandleFrontier}{$Frontier$}
\State recursively divide current frontier, atomically relax edges in frontier and building a new one

\State \textbf{return} $NewFrontier$
\EndProcedure

\end{algorithmic}
\end{algorithm}


\FloatBarrier
\subsection{Сравнение подходов}
Каждый из вышеизложенных подходов имеет свои особенности, что позволяет каждому из них конкурировать друг с другом на некоторых типах графов. Рассмотрим эти особенности.

С первого взгляда может показаться, что Алгоритм 3 имеет лишь одни недостатки - он имеет наименьшим образом по сравнению с последующими задействует все процессоры и при этом асимптотически равен остальным двум. Однако рассмотрим внимательнее каноничный аглоритм Беллмана-Форда и запустим его на плотном графе, где для каждого ребра верно, что индекс вершины источника меньше индекса вершины назначения. В этом случае каноничной версии достаточно будет сделать лишь две итерации внешнего цикла, поскольку на каждой итераций внутреннего цикла значение расстояния для текущей вершины будет корректно посчитано (очевидно доказывается по математической индукции). Так как количество итерации третьего алгоритма в подобных графах может быть значительным и размер текущей очереди может быть большим, а второму же алгоритму на таких графах неплохая способность параллелиться будет только вредить - она будет заметно увеличивать число итераций внешнего цикла. То есть в таких случаях последние два алгоритма работают хуже первого.   

Но очевидно, что в большинстве случаев последние два алгоритма будут показывать лучшие результаты. Сравним эти два подхода. Алгоритм, основанный на обходе в ширину, заметно сокращает количество вершин для обработки в пределах каждой итерации. Однако на плотных графах количество таких вершин значительно и такой подход показывает себя не с лучшей стороны. 

\FloatBarrier
\subsection{Тестирование}

Для подтверждения вышеприведенных замечаний все вышеизложенные подходы были реализованы на основе библиотеки для параллельных вычислений PASL. Тестирование производилось на ряде графов на машине 40-core Intel machine (with hyper-threading) with 4×2.4GHz Intel
10-core E7-8870 Xeon processors, a 1066MHz bus, and 256GB of
main memory. 

Алгоритмы тестировались на различных графовых структурах, которые перечислены в Таблице 1.1. При этом описание для формата приведено внизу таблицы.
\begin{table}
\centering

\begin{tabular}{c|c|c}  
Name & Vertices & Edges\\
\hline\hline
Complete TS & 7071 & 24995985 \\  
Complete sign & 3162 & 9995082  \\  
BalancedTree fraction & 8388607 & 8388608 \\  
SquareGrid sign & 2499561 & 4999122  \\  
RandomSparse fraction(0.5) sign & 2500000 & 25000000  \\  
RandomSparse fraction(0.96) sign(+) & 2500000 & 25000000  \\  
RandomDense fraction(0.5) sign & 5000 & 25000000  \\  
RandomDense fraction(0.96) sign(+) & 5000 & 25000000  \\  
\hline
\multicolumn{3}{l}{\footnotesize \textit{sign} - sign of weights on edges }\\
\multicolumn{3}{l}{\footnotesize \textit{fraction} - fraction of lexicographically sorted edges (edges of type X -> X+i) }\\
\multicolumn{3}{l}{\footnotesize \textit{TS} - exists only Lexicographically Sorted edges (fraction = 1) }\\
\multicolumn{3}{l}{\footnotesize }\\
\multicolumn{3}{l}{\footnotesize  expression \textit{RandomDense fraction(0.5) sign} means Random Dense }\\
\multicolumn{3}{l}{\footnotesize 	graph with specified fraction and any sign of weights}\\
\end{tabular}

\caption{Input graph description}
\label{bf_algo_comparison}
\end{table}

\begin{table}
\centering

\begin{tabular}{l|ccc|cc|cc|ccc|ccc}  
Algo №& \multicolumn{3}{c}{Complete} & \multicolumn{2}{c}{BalancedTree} & \multicolumn{2}{c}{SquareGrid} & \multicolumn{3}{c}{RandomSparse} & \multicolumn{3}{c}{RandomDense}\\
& TS & + & - & 0.5 & 1 & + & +- & 0.5+  & 0.5- & 0.96+ & 0.5+ & 0.5- & 0.96+\\
\hline\hline
3 & 2.43 & 4.65 & nc & 116.31 & 9.04 & 5.49 & 13.40 & nc & nc & 24.35 & nc & nc & 5.01 \\  
4 & 5.17 & 0.18 & 10.84 & 3.59 & 3.08 & 5.92 & 7.10 & 2.77 & 14.68 & 2.42 & 0.48  & 6.38  & 0.46 \\
5 & 44.63 & 0.37 & 23.55 & 0.44 & 0.31 & 4.42 & 0.58 & 0.98 & 22.59 & 0.76  & 0.60  & 10.25 & 0.71 \\
\hline
\end{tabular}

\caption{Bellman-Ford algorithms comparison}
\label{graph_description}
\end{table}

\FloatBarrier
\section{Выводы}

Из таблиц видно, что наши ожидания относительно применимости конкретных подходов оправдались. Первый из алгоритмов работает лучше на узком спектре графов с высоким fraction и высокой средней степенью вершины, второй хорошо работает на плотных графах и графах с отрицательным весом ребер, третий же заметно лучше остальных на разреженных графах.

Таким образом сделав простой анализ структуры графа мы сможем выбрать необходимый алгоритм для поиска кратчайшего пути. 
\FloatBarrier
