\startconclusionpage

Как было показано в данной работе, задача определение кратчайшего расстояния как от одной вершины, так и от множества может быть эффективна решена на нескольких вычислительных ядрах. Было предложено множество алгоритмов для нахождения расстояний, причем каждый из них имеет свои специфические особенности и рекомендации к применению на конкретных графах.

Параллельные алгоритмы из первой главы основаны на алгоритме Беллмана-Форда, который может быть использован в тех случаях, когда в графе присутствуют отрицательные ребра, циклы отрицательного веса или же когда заранее известно, что количество итерации будет невелико (как, например, в случае неориентированного невзвешенного графа алгоритм будет работать всего за $O(V + E)$). При этом, каждая из модификации оказалась лучше других на некоторых типа графах, что говорит о том, что после некоторого простого анализа входного графа мы можем выбрать наиболее оптимальный алгоритм для нашей ситуации.

В контексте параллельных алгоритмов поиска кратчайшего расстояния между каждой парой вершин был предложен алгоритм, идея которого была описана ранее (наивный алгоритм), а также предложена разработанная мною модификация для поиска расстояний на реальных социальных графах. Этот алгоритм оказался значительно быстрее наивной версии и может быть широко использован в подобных графов крупных социальных сетей, таких как Twitter, Facebook или Vkontakte. 

В предложенных алгоритмах использовались современные подходы и структуры данных для параллельных вычислений на графах (такие как Frontier), которые показали свою состоятельность и оказались заметно более эффективными по сравнению с предшествующими аналогами. Таким образом, есть все основания полагать, что подобные алгоритмы могут служить отличным эффективным решением для многопроцессорных архитектур с высоким количеством ядер.

\cite{FRONTIERSEARCH}
\cite{LIGRA}
\cite{CHUNKEDSEQ}
\cite{STANFORDGRAPHS}
\cite{CILK} 


\FloatBarrier
